%\subsection{Non-harmonicity}
%A series expansion of $\Sum_{n,m}V(R_{n}-R_{m+n})$ to higher orders than the quadratic term gives non-harmonic contributions. We look at the third-order term, which gives the first correction to the ideal boson gas picture.
%\[\dfrac{1}{3!}\Sum_{n,m}\underbrace{\dfrac{\partial^{3}V}{\partial x^{3}}\Bigg|_{x=R_{n}^{0}-R_{n+m}^{0}}}_{\Gamma(R_{n}^{0}-R_{n+m}^{0})=\Gamma(-R_{m}^{0})}\left(x_{n}-x_{n+m}\right)^{3}=\dfrac{1}{6}\Sum_{ %m}\Gamma(-R_{m}^{0})\Sum_{n}\left(x_{n}-x_{n+m}\right)^{3},\]
%where we used that $\Sum_{m}\Gamma(-R^{0}_{m})=\Sum_{m}\Gamma(R^{0}_{m})$.
%Again we introduce the Fourier transforms of $x_{n}$, such that
%\begin{align}
%	\Sum_{n}x_{n}^{3}=&\Sum_{n}\left(\dfrac{1}{\sqrt{N}}\Sum_{k_{1}}\tilde{x}_{k_{1}}e^{ik_{1}n}\right)\left(\dfrac{1}{\sqrt{N}}\Sum_{k_{2}}\tilde{x}_{k_{2}}e^{ik_{2}n}\right)\left(\dfrac{1}{\sqrt{N}}\Sum_{k_{3}}\tilde{x}_{k_{3}}e^{ik_{3}n}\right)\nonumber\\
%	=&\dfrac{1}{\sqrt{N}}\Sum_{k_{1},k_{2},k_{3}}\tilde{x}_{k_{1}}\tilde{x}_{k_{2}}\tilde{x}_{k_{3}}\underbrace{\dfrac{1}{N}\Sum_{n}e^{i(k_{1}+k_{2}+k_{3})n}}_{\delta_{k_{1},-k_{2}-k_{3}}}\nonumber \\
%	=&\dfrac{1}{\sqrt{N}}\Sum_{k_{1},k_{2}}\tilde{x}_{k_{1}}\tilde{x}_{k_{2}}\tilde{x}_{-k_{1}-k_{2}}\nonumber,
%\end{align}
while
\begin{align}
\Sum_{n}x_{n}^{2}x_{n+m}=&\dfrac{1}{\sqrt{N}}\Sum_{k_{1},k_{2}}\tilde{x}_{k_{1}}\tilde{x}_{k_{2}}\tilde{x}_{-k_{1}-k_{2}}e^{ik_{1}m},\nonumber
\end{align}
and by introducing $n'=n+m$ then
\begin{align}
	\Sum_{n}x_{n}x_{n+m}^{2}=&\Sum_{n'}x_{n'}^{2}x_{n'-m} \nonumber \\
	=&\dfrac{1}{\sqrt{N}}\Sum_{k_{1},k_{2}}\tilde{x}_{k_{1}}\tilde{x}_{k_{2}}\tilde{x}_{-k_{1}-k_{2}}e^{-ik_{1}m}.\nonumber
\end{align}
Now the non-harmonic third order term is
\[\begin{array}{r@{\;}l}
	\dfrac{1}{6}\Sum_{m}\Gamma(-R_{m}^{0})&\Sum_{n}(x_{n}-x_{n+m})^{3}\\\\
	& = \dfrac{1}{6}\Sum_{m}\Gamma(-R_{m}^{0})\Sum_{n}(x_{n}^{3}-x_{n+m}^{3}-3x_{n}^{2}x_{n+m}+3x_{n}x_{n+m}^{2})\\\\ 
	&=\dfrac{1}{6}\Sum_{m}\left(-\Gamma(-R_{m}^{0})\right)\Sum_{k_{1},k_{2}}\dfrac{1}{\sqrt{N}}\tilde{x}_{k_{1}}\tilde{x}_{k_{2}}\tilde{x}_{-k_{1}-k_{2}}\left(e^{ik_{1}m}-e^{-ik_{1}m}\right)\\\\
	&=\Sum_{k_{1},k_{2}}M(k_{1})\tilde{x}_{k_{1}}\tilde{x}_{k_{2}}\tilde{x}_{-k_{1}-k_{2}},
\end{array}\]
where
\begin{equation}
	M(k_{1})=\dfrac{-i}{3}\dfrac{1}{\sqrt{N}}\Sum_{m}\Gamma(-R_{m}^{0})\sin(k_{1}m).\nonumber
\end{equation}
The non-harmonic third order term can further be written
\[\begin{array}{r@{\;}l}
	\Sum_{k_{1},k_{2}} & M(k_{1})\left(\dfrac{\hbar}{2M}\right)^{\tfrac{3}{2}}\dfrac{1}{\sqrt{\omega_{k_{1}}\omega_{k_{2}}\omega_{k_{1}+k_{2}}}}\left(a^{\dagger}_{-k_{1}}+a_{k_{1}}\right)\left(a^{\dagger}_{-k_{2}}+a_{k_{2}}\right)\left(a^{\dagger}_{-k_{3}}+a_{k_{3}}\right)\\\\
	& = \Sum_{k_{1},k_{2}}G(k_{1},k_{2})\Big(a^{\dagger}_{-k_{1}}a^{\dagger}_{-k_{2}}a^{\dagger}_{-q}+a^{\dagger}_{-k_{1}}a^{\dagger}_{-k_{2}}a^{\dagger}_{q}+a^{\dagger}_{-k_{1}}a_{k_{2}}a^{\dagger}_{-q}+a^{\dagger}_{-k_{1}}a^{\dagger}_{k_{2}}a_{q}\\\\
	&\quad +a_{k_{1}}a^{\dagger}_{-k_{2}}a^{\dagger}_{-q}+a_{k_{1}}a^{\dagger}_{-k_{2}}a_{q}+a_{k_{1}}a_{k_{2}}a^{\dagger}_{-q}+a_{k_{1}}a_{k_{2}}a_{q}\Big)
\end{array}\]
It is possible to do similar calculations for even higher order non-harmonic terms. We notice that the number of phonons are not conserved in the scattering process resulting from the non-harmonic term. However; \ul{momentum is always conserved in all scattering processes!} Non-harmonic effects are most important in crystals with small atom masses, $M$, and large lattice constants.

We can generalize our method by looking at two- or three-dimensional lattices and introduce more than one mode, labelled by an index $\lambda$. For the \textit{harmonic} case we get
\[\Ha=\Sum_{\B{q},\lambda}\hbar\omega_{\B{q},\lambda}\Creab_{\B{q},\lambda}\Annib_{\B{q},\lambda}.\]
Now the oscillations are described by vectors:
\[\B{x}_{\B{q},\lambda} =\sqrt{\dfrac{\hbar}{2M\omega_{\B{q},\lambda}}}\B{\xi}_{\B{q},\lambda}\left(\Creab_{-\B{q},\lambda}+\Annib_{\B{q},\lambda}\right), \]
\[\B{p}_{\B{q},\lambda}=i\sqrt{\dfrac{\hbar M\omega_{\B{q},\lambda}}{2}}\B{\xi}_{\B{q},\lambda}\left(\Creab_{\B{q},\lambda}-\Annib_{-\B{q},\lambda}\right),   \]
where $\B{\xi}_{\B{q},\lambda}$ is the excitation direction for mode $\lambda$ with wave vector $\B{q}$.

\subsection{Electron-phonon coupling}
Up to now we have looked at the second quantized form of the electron gas and lattice vibrations. We now want to couple these degrees of freedom. 
\[\Ha_{\mathrm{phonon}}=\Sum_{\B{q},\lambda}\omega_{\B{q},\lambda}\Creab_{\B{q},\lambda}\Annib_{\B{q},\lambda},  \]
\[\Ha_{\mathrm{electron}}=\Sum_{\B{k},\sigma}\vareps_{\B{k}}\Crea_{\B{k},\sigma}\Anni_{\B{k},\sigma}+\dfrac{1}{2}\Sum_{\B{k},\B{k}',\B{q},\sigma,\sigma'}V(\B{q})\Crea_{\B{k},\sigma}\Crea_{\B{k}',\sigma'}\Anni_{\B{k}'+\B{q},\sigma'}\Anni_{\B{k}-\B{q},\sigma}  \]
Here we have used the plane wave representation for electrons, which is suitable for good metals. Earlier we looked at the second-quantized form of the one-particle operator 
\[H_{\mathrm{el-ion}} = \Sum_{i}U(\B{r}_{i})=\Sum_{i,j}V_{\mathrm{el-ion}}\left(\B{r}_{i};\B{R}_{j}\right)  \]
Previously we ignored the dynamics of the ion lattice when we second-quantized this term. When lattice excitations are included in $H_{\mathrm{el-ion}}$, this results in electron-phonon coupling.
\[V_{\mathrm{el-ion}}\left(\B{r}_{i};\B{R}_{j}\right)=V_{\mathrm{el-ion}}\left(\B{r}_{i}-\B{R}_{j}\right) \]
The equilibrium postion for lattice points are given by $\B{R}_{j}^{0}$, such that 
\[\B{R}_{j}=\B{R}_{j}^{0}-\Sum_{\lambda}\B{x}_{j,\lambda}.  \]
The minus sign is just a convention which gives less minus signs in the calculations. We now write $V_{\mathrm{el-ion}}$ in a Taylor series in terms of small lattice excitations.We only use the lowest order term which has a non-zero contribution, and this is expected to be a good approximation for \textit{weak electron-phonon coupling}. The series is
\[V_{\mathrm{el-ion}}(\B{r}_{i}-\B{R}_{j})=V_{\mathrm{el-ion}}(\B{r}_{i}-\B{R}_{j}^{0})+\Sum_{\lambda}\B{x}_{j,\lambda}\cdot\nabla V_{\mathrm{el-ion}}\Big|_{\B{r}_{i}-\B{R}_{j}^{0}}+... \] 
We have already looked at the first term, which results in
\[\Ha_{\mathrm{el-ion}}^{(1)}=\Sum_{\B{k},\B{q},\sigma}\tilde{U}(\B{q})\Crea_{\B{k}+\B{q},\sigma}\Anni_{\B{k},\sigma} \]
where $\tilde{U}(\B{q})$ is the Fourier transform of $U(\B{r})=\Sum_{j}V_{\mathrm{el-ion}}\left(\B{r}-\B{R}_{j}^{0}\right)$. This is the contribution from the \textit{static} ion lattice. We now look at
\[\Sum_{i,j,\lambda}\B{x}_{j,\lambda}\cdot\nabla V_{\mathrm{el-ion}}(\B{r}_{i}-\B{R}_{j}^{0}),\]
and by introducing
\[V_{\mathrm{el-ion}}(\B{r})=\dfrac{1}{\sqrt{N}}\Sum_{\B{q}}\tilde{V}_{\mathrm{e-i}}(\B{q})e^{i\B{q}\cdot\B{r}},\]
then
\[\nabla V_{\mathrm{el-ion}}=\dfrac{1}{\sqrt{N}}\Sum_{\B{q}}i\B{q} \tilde{V}_{\mathrm{e-i}}(\B{q})e^{i\B{q}\cdot\B{r}}.
\]
We want to find the second-quantized form of the second term
\begin{align}
H_{\mathrm{el-ion}}^{(2)}=&\dfrac{1}{\sqrt{N}} \Sum_{i,j,\lambda}\B{x}_{j,\lambda}\cdot\Sum_{\B{q}}i\B{q}\tilde{V}_{\mathrm{e-i}}(\B{q})e^{i\B{q}\cdot(\B{r}_{i}-\B{R}_{j}^{0})}  \nonumber \\
=& \Sum_{i}\Sum_{\B{q},\lambda}\left(\dfrac{1}{\sqrt{N}}\Sum_{j}\B{x}_{j,\lambda}e^{-i\B{q}\cdot\B{R}_{j}^{0}}\right)i\B{q}\tilde{V}_{\mathrm{e-i}}(\B{q})e^{i\B{q}\cdot\B{r}_{i}}\nonumber \\
=&\Sum_{i}\Sum_{\B{q},\lambda}\tilde{\B{x}}_{\B{q},\lambda}\cdot i\B{q}\tilde{V}_{\mathrm{e-i}}(\B{q})e^{i\B{q}\cdot\B{r}_{i}} \nonumber \\
\equiv&\Sum_{i}F(\B{r}_{i})\nonumber ,
\end{align}
where
\[F(\B{r}_{i})=\Sum_{q}\tilde{F}(\B{q})e^{i\B{q}\cdot\B{r}_{i}} \]
and 
\[\tilde{F}(\B{q})=\Sum_{\lambda}\B{x}_{\B{q},\lambda}\cdot i\B{q}\tilde{V}_{\mathrm{e-i}}(\B{q}).     \]
When $H_{\mathrm{el-ion}}^{(2)}=\Sum_{i}F(\B{r}_{i})$, we know that the second-quantized form in the plane wave basis is
\[\Ha_{\mathrm{el-ion}}^{(2)}=\Sum_{\B{k},\B{q},\sigma}\tilde{F}(\B{q})\Crea_{\B{k}+\B{q},\sigma}\Anni_{\B{k},\sigma},   \]
since $F(\B{r})$ is independent of spin. We now introduce the quantized phonon-gas:
\[\B{x}_{\B{q},\lambda}=\sqrt{\dfrac{\hbar}{2M\omega_{\B{q},\lambda}}}\B{\xi}_{\B{q},\lambda}\left(\Creab_{-\B{q},\lambda}+\Annib_{\B{q},\lambda}\right) ,   \]
such that 
\[\tilde{F}(\B{q})=\Sum_{\lambda}M_{\B{q},\lambda}\left(\Creab_{-\B{q},\lambda}+\Annib_{\B{q},\lambda}\right), \]
where
\begin{center}
\boxed{M_{\B{q},\lambda}=i\left(\B{q}\cdot\B{\xi}_{\B{q},\lambda}\right)\sqrt{\dfrac{\hbar}{2M\omega_{\B{q},\lambda}}}\tilde{V}_{\mathrm{e-i}}(\B{q})   }
\end{center}
Finally,
\begin{align}
	\Ha_{\mathrm{el-ion}}=&\Ha_{\mathrm{el-phonon}} \nonumber \\
	=&\Sum_{\B{k},\B{q},\sigma,\lambda}M_{\B{q},\lambda}\left(\Creab_{-\B{q},\lambda}+\Annib_{\B{q},\lambda}\right)\Crea_{\B{k}+\B{q},\sigma}\Anni_{\B{k},\sigma}.\nonumber
\end{align}
The quantity $M_{\B{q},\lambda}$ is the coupling constant between electrons and phonons. One important feature is that $M_{\B{q},\lambda}\propto\B{q}$, which means that
\begin{center}
	\boxed{\B{q}\rightarrow 0, \Rightarrow M_{\B{q},\lambda}\rightarrow 0.}
\end{center}
The case of $\B{q}=0$ is equivalent with that the whole lattice is translated uniformly, which does not affect the electrons due to Galilean invariance. Another important fact is that the coupling $M_{\B{q},\lambda}$ is independent of both $\B{k}$ and $\sigma$. The $\sigma$-independence comes naturally since the lattice potential is spin independent.  The coupling is independent of $k$ \Underline{only because we use the plane wave basis}. Also, $M_{\B{q},\lambda}\propto1/\sqrt{M}$, such that the coupling is weaker for heavy ions.
\section{Many-particle perturbation theory}
\subsection{Zero temperature}
Let us assume we can describe a system with a known Hamilton operator, for example
\[\begin{array}{r@{\;}c@{\;}l@{\qquad}l}
	\Ha_{0}^{\mathrm{F}}	& =	& \Sum_{\B{k},\sigma}\vareps_{\B{k}}\Crea_{\B{k},\sigma}\Anni_{\B{k},\sigma},				& \mathrm{(fermions)}\\\\
	\Ha_{0}^{\mathrm{B}}	& =	& \Sum_{\B{q},\lambda}\omega_{\B{q},\lambda}\Creab_{\B{q},\lambda}\Annib_{\B{q},\lambda},	& \mathrm{(bosons)}
\end{array}\]
and we want a description of the \textit{quantized} changes which happen as a result of perturbations of $\Ha_{0}$, i.e. $\Ha_{0}\rightarrow\Ha_{0}+V$. Usually, the perturbated Hamiltonian can not be diagonalized exactly. Examples of perturbations are:
\begin{enumerate}[i)]
	\item Electron-phonon coupling: \[V=\Sum_{\B{k},\B{q},\sigma,\lambda}M_{\B{q},\lambda}(\Creab_{-\B{q},\lambda}+\Annib_{\B{q},\lambda})\Crea_{\B{k}+\B{q},\sigma}\Anni_{\B{k},\sigma}\]
	\item Electron-electron interaction: \[V=\Sum_{\B{k},\B{k}',\B{q},\sigma,\sigma'}\tilde{V}(\B{q})\Crea_{\B{k},\sigma}\Crea_{\B{k}',\sigma'}\Anni_{\B{k}'+\B{q},\sigma'}\Anni_{\B{k}-\B{q},\sigma} \]
	\item The $U$-term in the Hubbard model.
\end{enumerate}
\subsection{Time-evolution of states}
Hamilton-operator \[\Ha=\Ha_{0}+V.\]
$\Ha_{0}$: One-particle operator. 
$V$: Perturbation, such that we can not solve the problem exact.
\begin{Indentskip}
	\subsubsection*{Schr\"odinger picture}
	\Underline{Operators} are independent of time: $\hat{\mathcal{O}}(t)=\hat{\mathcal{O}}(0)$.\\
	\Underline{States} are time-dependent: \[i\dfrac{\partial }{\partial t}|\psi\rangle =\Ha|\psi\rangle \]
	Formally: \[|\psi(t)\rangle=e^{-i\Ha t}|\psi(0)\rangle  \] 
	$e^{-i\Ha t}$: Evolution operator which takes the system from the state $|\psi(0)\rangle$ at time $t=0$ to the state $|\psi(t)\rangle$ at time $t$.
\end{Indentskip}
\begin{Indentskip}
	\subsubsection*{Heisenberg picture}
	\Underline{States} are time-independent: $|\psi(t)\rangle=|\psi(0)\rangle$.\\
	\Underline{Operators} are time-dependent: \[\dfrac{d\hat{\mathcal{O}}}{dt} =-i\left[\hat{\mathcal{O}},\Ha\right]  \] 
	Formally:
	\[\hat{\mathcal{O}}(t)=e^{i\Ha t}\hat{\mathcal{O}}(0)e^{-i\Ha t}\] 
	Schrödinger: \[\Braket{\psi(t)|\hat{\mathcal{O}}(0)|\psi(t)}=\Braket{\psi(0)|e^{i\Ha t}\hat{\mathcal{O}}(0)e^{-i\Ha t}|\psi(0)}.\]
	Heisenberg: \[\Braket{psi(0)|\hat{\mathcal{O}}(t)|\psi(0)}=\Braket{\psi(0)|e^{i\Ha t}\hat{\mathcal{O}}(0)e^{-i\Ha t}|\psi(0)}\]	
	We get the same matrix elements in both pictures! 
\end{Indentskip}
There is a third picture, which is used in perturbation theory.

\begin{Indentskip}
	\subsubsection*{Interaction picture}
	Some time dependence in \textit{both} the states and the operators.
	\[|\psi(t)\rangle=e^{i\Ha_{0}t}e^{-i\Ha t}|\psi(0)\rangle,\]
	\[\hat{\mathcal{O}}(t)=e^{i\Ha_{0}t}\hat{\mathcal{O}}(0)e^{-i\Ha_{0}t}.\]
	
\end{Indentskip}
\Underline{NB:} In general, $\left[\Ha_{0},V\right]\neq0$, such that $e^{i\Ha_{0}t}e^{-i\Ha t}$ is much more complicated than $e^{-iVt}$!

For $V=0$ the interaction picture coincides with the Heisenberg picture. Since we assume that the components of $V$ is much smaller than the components of $\Ha_{0}$ for the perturbation theory to work, it follows that most of the time development lies in the operators, while the states has a slow time-variation. 

\textbf{\Underline{Evolution operator for a state $|\psi(t)\rangle$ in the interaction picture:}}
\begin{align}
	|\psi(t)\rangle &= U(t)|\psi(0)\rangle,\nonumber \\
U(t)&=e^{i\Ha_{0}t}e^{-i\Ha t}\neq e^{i(\Ha_{0}-\Ha_{0}-V)t} ~~(!!)\nonumber
\end{align}

\boxed{V=0 : \\
U(t)=. \left[\Ha_{0},\Ha_{0}\right]=.  e^{i\Ha_{0}t}e^{-i\Ha_{0}t}=1. } 

$U(t)$ have a non-trivial time-dependence only when $V\neq0$, and we want to find $U$ as a functional of $V$. (A functional is a number which depends on a function, i.e. a mapping from a space of functions onto the complex numbers.) We find $U(t)$ from $V(t)$ like this: 
\begin{align}
	\dfrac{\partial U}{\partial t} =&~~ \dfrac{\partial}{\partial t}\left(e^{i\Ha_{0}t}e^{-i\Ha t}\right)\nonumber \\
	=&~~ i\Ha_{0}e^{i\Ha_{0}t}e^{-i\Ha t}-e^{i\Ha_{0}t}i\Ha e^{-i\Ha t} \nonumber \\
	=&~~ ie^{i\Ha_{0}t}\left(\Ha_{0}-\Ha \right)e^{-i\Ha t} \nonumber \\
	=&~~ ie^{i\Ha_{0}t}Ve^{-i\Ha t}\nonumber \\ 
	=& -i\underbrace{e^{i\Ha_{0}t}Ve^{-i\Ha_{0}t}}_{\hat{V}(t)}\underbrace{e^{i\Ha_{0}t}e^{-i\Ha t}}_{U(t)}\nonumber \\
	=&-i\hat{V}(t)U(t).\nonumber 
\end{align}
$\hat{V}(t)$ explisitly time-dependent, such that $U(t)\neq e^{-i\hat{V}t}U(0)$. When $V=0$ then $\partial U/\partial t=0$, such that $U(t)=U(0)=1$. We integrate over $t$:
\begin{align}
	\int_{0}^{t}dt'\dfrac{\partial U(t')}{\partial t'}&=U(t)-U(0)\nonumber \\ 
	&= -i\int_{0}^{t}dt'\hat{V}(t')U(t'),
\end{align}
such that 
\begin{center}
		\boxed{U(t)=1-i\int_{0}^{t}dt'\hat{V}(t')U(t')}
			\end{center}
\vspace{5pt}
This is a (linear) integral equation for $U(t)$. We want to solve this by iteration, but first we introduce a more general evolution operator.



\subsection{The $S$-matrix}
The scattering matrix, $S$, is the central quantity in many-particle perturbation theory. With a perturbation theory for $S$ then we will almost automatically have a method to write a perturbation series for \textbf{\Underline{any physical quantity}}. The $S$-matrix is defined as:
\[|\psi(t)\rangle=S(t,t')|\psi(t')\rangle,\] such that
$S(t,0)=U(t),$ by recalling that $|\psi(t)\rangle = U(t)|\psi(0)\rangle$. The state can be written as
\begin{align}
	|\psi(t)\rangle&=S(t,t')|\psi(t')\rangle\nonumber \\
	&=S(t,t')U(t')|\psi(0)\rangle, \nonumber
\end{align}
such that 
\[U(t)=S(t,t')U(t'),\]
or in other words
\[S(t,t')=U(t)U^{-1}(t').\]
We utilize the unitarity of $U$, which we find from
\[U^{-1}(t)=e^{i\Ha t}e^{-i\Ha_{0}t}=U^{\dagger}(t),\] 
such that the scattering matrix is written as
\begin{center}
	\boxed{S(t,t')=U(t)U^{\dagger}(t').}
\end{center}
We want to find an integral equation for $S$:
\begin{align}
	\dfrac{\partial S(t,t')}{\partial t} &= \dfrac{\partial U(t)}{\partial t}U^{\dagger}(t')\nonumber \\
	&= -i\hat{V}(t)U(t)U^{\dagger}(t')\nonumber \\
	&= -i\hat{V}(t)S(t,t'). \nonumber
\end{align}
We integrate over time: 
\begin{align}
	\int_{\tilde{t}}^{t}\dfrac{\partial S(t'',t')}{\partial t''} =& S(t,t')-S(\tilde{t},t') \nonumber \\
	=&-i\int_{\tilde{t}}^{t}dt''\hat{V}(t'')S(t'',t').
\end{align}
In the special case of $\tilde{t}=t'$, then $S(t',t')=1=U(t')U^{\dagger}(t')$. The integral equation for the scattering matrix then follows:
\begin{center}
	\boxed{S(t,t')=1-i\int_{t'}^{t}dt''\hat{V}(t'')S(t'',t')}
\end{center}
The time evolution of $S$ is given by $\hat{V}(t)=e^{i\Ha_{0}t}Ve^{-i\Ha_{0}t}$, and in a similar way as for $U(t)$ we can construct a perturbation theory for $S(t,t')$ by solving the integral equation by iteration.

\Underline{0'th approx.:} 
\[S_{0}(t,t')=1.\]
\Underline{1st approx.:}
\[S_{1}(t,t')=1-i\int_{t'}^{t}dt''\hat{V}(t'').\]
Now we can see what it means that $V$ should be small, i.e. \[\left|\int_{t'}^{t}dt''\hat{V}(t'')\right|\ll1.\]
\begin{Indentskip}
	\subsubsection*{Properties of the $S$-matrix:}
	\[S(t,t)=1,\]
	\begin{align}
		\left(S(t,t')\right)^{\dagger}&=\left(U(t)U^{\dagger}(t')\right)^{\dagger} = U(t')U^{\dagger}(t)\nonumber \\
	&=S(t't),\nonumber
	\end{align}
	\[S(t,t'')=S(t,t')S(t',t''),\] since
	\begin{align}
		|\psi(t)\rangle&=    \nonumber
	\end{align}
\end{Indentskip}
To find an exact expression for $S$ we introduce the \Underline{time ordering operator}: $\tilde{T}$.




\subsection{Time ordering of operators}
\Underline{Bosons:}
Two boson operators have time-ordered product 
\[\tilde{T}[A(t_{1})B(t_{2})]= \begin{cases}A(t_{1})B(t_{t}); & t_{1}>t_{2}.\\
B(t_{2})A(t_{1}); & t_{2}>t_{1}.
\end{cases} \]

\Underline{Fermions:}
Two fermion operators have time-ordered product
\[\tilde{T}[A(t_{1})B(t_{2})]= \begin{cases}A(t_{1})B(t_{t}); & t_{1}>t_{2}.\\
-B(t_{2})A(t_{1}); & t_{2}>t_{1}.
\end{cases} \] 
The operator $\tilde{T}$ orders the operator with the earliest time to the right, such that it operates first on a ket, $|\psi\rangle$.

We look further at the equation for $S$:
\[S(t,t')=1-i\int_{t'}^{t}dt''\hat{V}(t'')S(t'',t'),\]
\Underline{0.:}
\[S_{0}(t,t')=1,\]
\Underline{1.:}
\[S_{1}(t,t')=1-i\int_{t'}^{t}dt''\hat{V}(t''),\]
\Underline{2.:}
\begin{align}
	S_{2}(t,t')&=1-i\int_{t'}^{t}dt''\hat{V}(t'')S_{1}(t'',t')\nonumber \\
	&=1-i\int_{t'}^{t}dt''\hat{V}(t'')\left(1-i\int_{t'}^{t''}dt'''\hat{V}(t''')\right) \nonumber \\ 
	&=1+(-i)\int_{t'}^{t}dt''\hat{V}(t'')+(-i)^{2}\int_{t'}^{t}dt''\int_{t'}^{t''}dt'''\hat{V}(t'')\hat{V}(t'''),\nonumber
\end{align}
This can be done infinitly, such that
\[S(t,t')=1+\Sum_{n=1}^{\infty}(-i)^{n}\int_{t'}^{t}dt_{1}\int_{t'}^{t_{1}}dt_{2}...\int_{t'}^{t_{n-1}}dt_{n}\left(\hat{V}(t_{1})\hat{V}(t_{2})...\hat{V}(t_{n})\right).\]
This expression is such that all the integrations have the same lower bound, $t'$, while the upper bounds are different. We use time ordering of the operators to get all the upper integration bounds the same. Let us consider a $\hat{V}$ consisting of a combination of fermion and/or boson operators in a way such that 
\[\tilde{T}\left[V(t_{1})V(t_{2})\right]=\begin{cases}
V(t_{1})V(t_{2}); & t_{1}>t_{2} \\
V(t_{2})V(t_{1}); & t_{2}>t_{1}. \end{cases}\]
We then look at the expression
\begin{align}
	\dfrac{1}{2!}\int_{t'}^{t}dt_{1}\int_{t'}^{t}dt_{2}\tilde{T}\left[\hat{V}(t_{1})\hat{V}(t_{2})\right] =&~~~ \dfrac{1}{2}\int_{t'}^{t}dt_{1}\int_{t'}^{t_{1}}dt_{2}\hat{V}(t_{1})\hat{V}(t_{2}) \nonumber \\
	&+\dfrac{1}{2}\int_{t'}^{t}dt_{2}\int_{t'}^{t_{2}}dt_{1}\hat{V}(t_{2})\hat{V}(t_{1}) \nonumber \\
	=& \int_{t'}^{t}dt_{1}\int_{t'}^{t_{1}}dt_{2}\hat{V}(t_{1})\hat{V}(t_{2}),
\end{align}
by interchanging the labels $t_{1}\leftrightarrow t_{2}$ in the second term. In the same way:
\[\dfrac{1}{n!}\int_{t'}^{t}dt_{1}\int_{t'}^{t}dt_{2}...\int_{t'}^{t}dt_{n}\tilde{T}\left[\hat{V}(t_{1})...\hat{V}(t_{n})\right]=\int_{t'}^{t}dt_{1}\int_{t'}^{t_{1}}dt_{2}...\int_{t'}^{t_{n-1}}dt_{n}\hat{V}(t_{1})...\hat{V}(t_{n}).\]
We then get an expression for $S$:
\begin{align}
	S(t,t')&=1+\Sum_{n=1}^{\infty}\dfrac{(-i)^{n}}{n!}\int_{t'}^{t}dt_{1}...\int_{t'}^{t}dt_{n}\tilde{T}\left[\hat{V}(t_{1})...\hat{V}(t_{n})\right]\nonumber \\
	&=1+\tilde{T}\Sum_{n=1}^{\infty}\dfrac{(-1)^{n}}{n!}\left[\int_{t'}^{t}dt''\hat{V}(t'')\right]^{n},
\end{align}
which can be written formally as 
\begin{center}
	\boxed{S(t,t')=\tilde{T}\left[\exp\left(-i\int_{t'}^{t}dt''\hat{V}(t'')\right)\right]}
\end{center}
	Typically, we want to calculate expectation values on the form
	\[\langle\psi(0)|\hat{O}(t)|\psi(0)\rangle.\]
	Here, the state $|\psi(0)\rangle$ is the exact ground state for the interacting system. The problem is that $|\psi(0)\rangle$ is unknown, since $\Ha=\Ha_{0}+V$ is such that we cannot solve the problem exact. Can we relate $|\psi(0)\rangle$ formally to the groundstate $|\phi\rangle_{0}$ of $\Ha_{0}$? 
	
	The Gell-Mann-Low relations solves this. The trick is to introduce the perturbation $V$ adiabatically, i.e. the potential is turned on really slowly. We can use 
	\[\Ha = \Ha_{0}+Ve^{-|t|\epsilon},\]
	where $\epsilon$ is very small, and we assume that we study the system for times $|t|\ll\epsilon$.