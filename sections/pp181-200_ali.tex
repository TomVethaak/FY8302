\clearpage



\begin{Indent}
  Fermi liquids are characterized by having $Z_{\bm k} > 0$ at the Fermi level!
\end{Indent}
It is a bit complicated to calculate~$\Sigma$ (see the integral expression in a previous chapter).
We will not perform the calculations here, but note that when the perturbation is the Coulomb potential, then the result is
\[
  \Sigma_I = -A_{\bm k}\, \mathrm{sgn}(\omega-\epsilon_F) (\omega-\epsilon_F)^2 = \Sigma_I(\bm k, \omega) \,,
\]
where $A_{\bm k} > 0$.
When $\omega \rightarrow \epsilon_F$, then $\Sigma_I \rightarrow 0$ as $(\omega-\epsilon_F)^2$:

\begin{figure}[H]
  \centering
  \includegraphics[width=0.5\textwidth]{img/pp181-200_selfenergy.pdf}
\end{figure}

The quasiparticles are well-defined close to the Fermi level because the damping is small.
If we get $\Sigma_I \sim (\omega-\epsilon_F)^2$, $\alpha < 1$, the damping disappears too slow, and $\alpha < 1$ is an indication that this is \emph{not} a Fermi liquid.
Finding a theoretical foundation for \emph{non}-Fermi liquid theory in $D>1$ dimensions is one of the big challenges in theoretical physics today (1995), since the normal ($T>T_c$) metallic state of high-$T_c$ superconductors exhibit transport properties that cannot be described as a Fermi liquid.
This may possibly be related to two-dimensional physics.

\emph{Heavy-fermion systems} (e.g. \ce{UPt3}): Correlation (Coulomb) effects are so large that the quasiparticles have masses $m^x/m \sim 1000$, and yet we have $Z_{k_F} > 0$!

3D: Fermi liquid theory is incredibly robust.

2D: ???

1D: Does not work, there is new physics here.



\clearpage
\section{Superconductivity}
\subsection{Introduction}
The perturbation theory we have considered so far is well equipped to describe \emph{quantitative} changes in e.g. electron systems caused by many-particle effects.
Pure perturbation theory is however not well suited to describe qualitative changes in a fermion or boson system caused by many-particle effects.
Examples of such qualitative changes may be:
\begin{enumerate}[(i)]
  \item Melting;
  \item Bose--Einstein condensation;
  \item Metal $\leftrightarrow$ insulator transitions;
  \item Metal/insulator  $\leftrightarrow$ superconductor transitions;
  \item Paramagnetic $\leftrightarrow$ ferromagnetic/antiferromagnetic transitions;
  \item Uniform electron gas $\leftrightarrow$ electron lattice (Wigner crystal) at low electron densities.
\end{enumerate}
Let $H = H_0 + V$, where $H_0$ for instance may describe a simple metal in in case~(iv) or a paramagnet in case~(v), and $V$ represents a perturbation that we expect to qualitatively change the ground state of $H_0$.

Absolutely \emph{all} transitions from one phase to another (i.e. phase transition)---such as e.g. the transition from a lattice to a fluid in the first example---are characterized by:

\begin{Indent}
  It is \emph{impossible} to reach one phase from a different phase by using a finite-order perturbation expansion in $V$.
\end{Indent}

If we wish to have any hope of describing transitions of type (i--vi) above, we need to do something different than pure perturbation theory.
We have already seen one example of this in problem set~1 (problem~2), where we replaced a many-particle problem by a selfconsistent one-particle problem that was exactly solvable.
In that case, we introduced a variational parameter~$\Delta$, which we determined by minimizing the energy with respect to~$\Delta$.
We will take a similar approach for the metal$\leftrightarrow$superconductor transition.
In problem set~1 (problem~2) the instability we considered was of type~(v), i.e. a transition from a non-magnetic to an antiferromagnetic state.
We saw that $\Delta\neq0$ signalized an \emph{ordering of spins}.
The model we considered was the Hubbard model,
\[
  H = \sum_{\langle i,j \rangle \sigma} t c^\dagger_{i\sigma} c_{j\sigma} + u\sum_i n_{i\uparrow} n_{i\downarrow} \,.
\]
We calculated non-perturbatively in $u$, and found that:
\[
  \Delta \sim e^{-1/Du} \,.
\]
NOTE: $\Delta$ is a non-analytic function of $u$.
This signalizes that the answer cannot be reached by any finite-order perturbation theory in $u$, because $\Delta$ is \emph{not} of the form
\[
  \Delta = \sum_{n=0}^\infty c_n u^n\,.
\]



\subsection{The Cooper Problem}
We have previously seen that it is possible for electrons to experience attractive interactions via phonons.
Before we attempt to solve this many-particle problem, we will first consider the consequences of electron--electron attraction in a much simpler model; the math will illustrate that the final result will not be attainable by a finite-order perturbation expansion in this case either.

The model we will use is as follows: Fermi sphere $\ket{\textrm{FS}}$ + two extra electrons at \emph{opposite sides} of $\ket{\textrm{FS}}$ with \emph{opposite spins}.
\begin{enumerate}[(i)]
  \item Electrons in $\ket{\textrm{FS}}$ do \emph{not} interact with each other;
  \item The extra electrons do not interact with $\ket{\textrm{FS}}$ except via the Pauli principle: $k>k_F$;
  \item The extra electrons are in the one-particle states $\ket{\bm k,\sigma}$ and $\ket{-\bm k,-\sigma}$;
  \item The two electrons interact attractively in a thin ``shell'' around the Fermi surface.
\end{enumerate}
\begin{figure}[H]
  \centering
  \includegraphics[width=0.5\textwidth]{img/pp181-200_fermisurface.pdf}
\end{figure}
The typical Debye frequency in a lattice is $\omega_D \sim 100$~K (since we assume a phonon-mediated attraction).

\clearpage
We will approximate the effective potential $\tilde{V}_\textrm{eff}$ by a step function:
\begin{figure}[H]
  \centering
  \includegraphics[width=\textwidth]{img/pp181-200_veffapprox.pdf}
\end{figure}
If at least one of the two electrons are outside of the thin shell around the Fermi level, they do not ``see'' each other.

We will solve this two-particle pproblem \emph{exactly}.
This will also provide a basis for discussing how well perturbation theory possibly could work.
\[
  H = H_0 + \tilde{V}_\textrm{eff}
\]
$H_0$: Kinetic energy of the two extra electrons.  \\
Exact two-particle state: $\ket{1,2}$. \\
If the electrons did not interact, then we would have the problem:
\[
  H_0 \ket{1,2}_0 = \epsilon_{\bm k} \ket{1,2}\,,
\]
which has the solution $\ket{1,2}_0 = \ket{\bm k,-\bm k}$.
$\epsilon_{\bm k}$: the \emph{kinetic} energy of the two extra electrons.

We now have to solve the problem:
\[
  (H_0 + \tilde{V}_\textrm{eff}) \ket{1,2} = E\ket{1,2} \,.
\]
From this Schrödinger equation we wish to find $\ket{1,2}$ and $E$.
$E$ is the energy of the \emph{interacting} two-electron system.

We start by expanding $\ket{1,2}$ in $\ket{1,2}_0 = \ket{\bm k,-\bm k}$:
\[
  \ket{1,2} = \sum_{\bm k'} a_{\bm k'} \ket{\bm k',-\bm k'}
\]
NOTE: $\bm k': \epsilon_{\bm k'} > 2\epsilon_F$ (Pauli principle, $\epsilon_{\bm k}$ two-electron energy) 

Substitute this into the Schrödinger equation:
\[
  \begin{aligned}
       \sum_{\bm k'} a_{\bm k'} \big[ H_0\ket{\bm k', -\bm k'} + \tilde{V}_\textrm{eff}\ket{\bm k', -\bm k'} \big]
    &= \sum_{\bm k'} Ea_{\bm k'} \ket{\bm k', -\bm k'} \big]  \\
    &\Downarrow \\
       \sum_{\bm k'} a_{\bm k'} \big[ E-\epsilon_{\bm k'} \big] \ket{\bm k', -\bm k'}  
    &= \sum_{\bm k'} a_{\bm k'} \tilde{V}_\textrm{eff}\ket{\bm k', -\bm k'} \big]
  \end{aligned}
\]
If we multiply by $\bra{\bm k,-\bm k}$ and use $\braket{\bm k,-\bm k|\bm k',-\bm k'} = \delta_{\bm k,\bm k'}$, we get an equation for determining $a_{\bm k}$ and $E$:
\[
  a_{\bm k}(\epsilon_{\bm k}-E) = -\sum_{\bm k'} a_{\bm k'} \braket{\bm k,-\bm k|\tilde{V}_\textrm{eff}|\bm k',-\bm k'} \equiv -\sum_{\bm k'} a_{\bm k'} V_{\bm k,\bm k'} \,,
\]
where the sum only extends over the values of $\bm k'$ that satisfy $\epsilon_{\bm k'} > 2\epsilon_F$.
The matrix element $V_{\bm k,\bm k'}$ describes the scattering process:
\begin{figure}[H]
  \centering
  \includegraphics[width=0.45\textwidth]{img/pp181-200_coopermodel.pdf}
\end{figure}
\[
  V_{\bm k,\bm k'} =
  \begin{cases}
              -  V\,, & |\epsilon_{\bm k}-2\epsilon_F|, |\epsilon_{\bm k'}-2\epsilon_F|<2\omega_D\,; \\
     \phantom{-} 0\,, & \text{otherwise}.
  \end{cases}
\]
We then obtain
\[
  a_{\bm k}(\epsilon_{\bm k}-E) = V\sum_{\bm k'} \theta(2\omega_D - |\epsilon_{\bm k'} - 2\epsilon_F|) \,,
\]
and switch to an energy integration
\[
  a(\epsilon)(\epsilon-E) = V\int_{2\epsilon_F}^{2\epsilon_F+2\omega_D} \mathrm{d}\epsilon'\; a(\epsilon') N(\epsilon') \,,
\]
where $N(\epsilon')$ is the density of states.
We integrate over a thin shell around the Fermi level.
If we now assume that $N(\epsilon)$ varies slowly around the Fermi level, we may set $N(\epsilon) \approx N(2\epsilon_F)$.
We introduce $\lambda \equiv VN(2\epsilon_F)$:
\[
  a(\epsilon)(\epsilon-E) = \lambda \int_{2\epsilon_F}^{2\epsilon_F+2\omega_D} \mathrm{d}\epsilon'\; a(\epsilon') \,.
\]
The integral is independent of $\epsilon$!
This means that $a(\epsilon)(\epsilon-E) = \text{const}$, so:
\[
  a(\epsilon) = \frac{\text{const}}{\epsilon-E} \,.
\]
Thus, the expansion coefficients have been determined.

Now, what remains is to find the energy~$E$.
Substitute the ansatz $a(\epsilon) = \text{const}/(\epsilon-E)$ into the previous equation, 
\[
  \text{const} = \lambda \int_{2\epsilon_F}^{2\epsilon_F+2\omega_D} \mathrm{d}\epsilon'\; \frac{\text{const}}{\epsilon'-E} \,.
\]
This equation determines the eigenvalue $E$:
\[
  \frac{1}{\lambda} = \ln\left|\frac{2\epsilon_F+2\omega_D-E}{2\epsilon_F-E}\right|
\]
Introduce $\Delta \equiv 2\epsilon_F - E$, where $\Delta$ is the binding energy for the interacting two-electron system relative to free electrons at the Fermi level
\[
  \frac{1}{\lambda} = \ln\left(1+\frac{2\omega_D}{\Delta}\right) > 0 \,.
\]
Can only be solved if $\lambda > 0$: $V$ \emph{attractive}.
\[
  \Delta = 2\omega_D \frac{1}{e^{1/\lambda}-1}
\]
When $\lambda \ll 1$:
\[
  \Delta \approx 2\omega_D e^{-1/\lambda} \,.
\]

$\Delta$: Not an analytic function of $\lambda$.
Note the similarity with the answer from problem set~1, even if $\Delta$ now refers to something completely different.
\[
  \Delta \neq \sum_{n=0}^\infty b_n \lambda^n \,:
\]
We do not have any Taylor expansion in $\lambda$.
The form of the answer again indicates that binding between two electrons that are attracted to each other is a non-perturbative effect.

\begin{Indent}
  $\Delta>0 \Rightarrow$ bound state of two electrons $\ket{+\bm k,+\sigma}$ and $\ket{-\bm k,-\sigma}$. \\
  The bound state is called a \emph{Cooper pair}.
\end{Indent}

A few comments:
\begin{enumerate}[(i)]
  \item A Cooper pair is a paired state in $\bm k$-space, not in $\bm r$-space.
        Even if a ``hydrogen atom'' image of the Cooper pair some times can be useful, the $\bm r$-space analogy should not be taken too far.
  \item Even if we have explicitly calculated for only \emph{two} electrons, it is still hidden a kind of many-particle effect here via the Pauli principle.
        $\Delta \sim e^{-1/\lambda}$, $\lambda = 2VN\epsilon_F$. \\
        $\epsilon_k \sim k^2 \Rightarrow N(\epsilon) \sim \epsilon^{1/2}$ in $d=3$\\
        $\epsilon_F \rightarrow 0 \Rightarrow 2N\epsilon_F \rightarrow 0 \Rightarrow \lambda \rightarrow 0 \Rightarrow \Delta \rightarrow 0$\,.
        It is therefore necessary to have a Fermi sea in the problem to obtain the bound state.
        Two electrons in vacuum that interacted attractively would not form a Cooper pair.
  \item \emph{Quantum effect}: substitute back the Planck constant: $\Delta = 2\hbar \omega_D e^{1\lambda}$. 
        In the classical limit $\hbar \rightarrow 0$, we also get $\Delta \rightarrow 0$.
        A classical electron gas with attractive interactions do not yield Cooper pairs.
        Can never explain Cooper pairs from Newtons equations.
  \item \emph{Temperature effect}: If we increase $T$, we will get thermal dissociation of Cooper pairs.
        We expect that all pairs are dissociated at a temperature such that $T \sim \Delta$.
\end{enumerate}



\clearpage
\begin{figure}[H]
  \centering
  \includegraphics[width=0.5\textwidth]{img/pp181-200_cooperlimits.pdf}
\end{figure}
